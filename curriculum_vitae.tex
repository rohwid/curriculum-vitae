\documentclass[11pt,a4paper]{moderncv}
\usepackage[top=1.5cm, bottom=1.5cm, left=1.0cm, right=1.0cm]{geometry}
\usepackage{moderncvheadi}
\usepackage{moderncvbodyi}
\usepackage[utf8]{inputenc}
\usepackage{pgf}
\usepackage{graphicx}
\usepackage{multibib}
\newcites{book,misc}{{Books},{Others}}


% set leftside document info length by:
\AtBeginDocument{\settowidth{\hintscolumnwidth}{Marital Status}}


% change CV color 
% red, green, blue, purple, grey, black
\moderncvcolor{darkgrey}


% title or possition / file name in overleaf
%\title{curriculum-vitae} 


\firstname{\Huge Nur Rohman}
\familyname{\Huge Widiyanto}
\address{Jl. Semolowaru Indah Blok E/8}{Dist. Sukolilo, Surabaya (60119)}{}   
\mobile{+62~857~319~25725}                                                 
\email{rohwid@gmail.com}
\social[linkedin][https://www.linkedin.com/in/rohwid/]{linkedin.com/in/rohwid}
\social[github][https://github.com/rohwid/]{github.com/rohwid}
\photo[65pt]{img/foto3.jpg}

\makeatletter\renewcommand*{\bibliographyitemlabel}{\@biblabel{\arabic{enumiv}}}\makeatother

                
\begin{document}


\makecvtitle


\section*{Summary}
\cvline{}{A lifelong learner who has a big passion for information technologies. I have excellent technical skills and an academic background in engineering (software and computer network), statistics, and machine learning. I also have experience as a Lead Network Engineer at a startup company previously. Specializing in software development, artificial intelligence, distributed system, cloud and grid computing. With a can do attitude, I ready to accepts new challenge, try to adapt quickly, and strive to give real impact to the organization.}




\section*{Basic Information}
\cvline{Gender}{Male}
\cvline{Date of Birth}{26\textsuperscript{th} March 1991}
\cvline{Place of Birth}{Lamongan}
\cvline{Marital Status}{Single}
\cvline{Nationality}{Indonesia}
\cvline{Languages}{Indonesian (Native), English (Fluent).}
%\cvline{Competencies}{Software Development/Engineering, Project Management, Team Building/Leadership.}
%\cvline{Interests}{Games, Movies, Books, Arts, Otomotive, Technology, Startup/Entrepreneurship, Organization/Management.}




\section*{Professional Experience}
%\cventry{June 2020 -- Present}{Senior Programmer}{PT. Nandar Baya Aerlangga}{}{Surabaya}{Handled backend program, create API, server deployment, and manage front-end programmers (Web and Android).}

\cventry{July 2019 -- July 2019}{Assistant Trainer}{Digital Talent Scholarship 2019 FGA (Fresh Graduate Academy)}{Institut Teknologi Sepuluh Nopember \& Ministry of Communication and Information Technology}{Surabaya}{Assisted the main trainer to deliver courses about the basics of Python for Artificial Intelligence in AI class and MongoDB databases in IoT class.}

\cventry{Nov 2018 -- Nov 2018}{Assistant Trainer}{Digital Talent Scholarship 2018}{Institut Teknologi Sepuluh Nopember \& Ministry of Communication and Information Technology}{Surabaya}{Assisted the main trainer to deliver courses about NLP (Natural Language Processing).}

\cventry{Sept 2017 -- Mar 2019}{Research Assistant}{Telematics Laboratory (B. 201)}{Institut Teknologi Sepuluh Nopember}{Surabaya}{Helped the lecturers to research program and teaching activity, handled network operations and servers in the laboratory.}

\cventry{Sept 2014 -- Aug 2016}{Lead Network Engineer}{PT. Niltava Teknologi Indonesia}{Surabaya}{}{Handled network operations, internet connection troubleshooting, backend program operation and deployment on cloud servers (AWS EC2), setup of local testing server with KVM. Jujucharm from Ubuntu was used to deploy backend program in multiple servers.}

\cventry{Jan 2013 -- Feb 2013}{Student Internship}{PT. Equnix Business Solution}{Jakarta}{}{Created software to monitor and control the network with C socket programming as back-end process and web-based interface.}

\cventry{Feb 2012 -- Mar 2013}{Assistant Coordinator}{B. 201 (Telematics Laboratory) and other Computer Engineering and Telematics Laboratories}{Institut Teknologi Sepuluh Nopember}{Surabaya}{Manage and coordinate research activity from laboratory assistant members or development groups, laboratory events or training, and laboratory sustainability (Housekeeping, server and internet connection maintenance).}

\cventry{Sept 2011 -- Sept 2015}{Laboratory Assistant}{B. 201 (Telematics Laboratory) and other Computer Engineering and Telematics Laboratories}{Institut Teknologi Sepuluh Nopember}{Surabaya}{Became laboratory assistant member (Basic Programming and Digital Circuit), conduct research, held events (training and exhibition), housekeeping and maintain laboratory equipment and sustainability (Servers, Tools, and Internet Connection).}




\section*{Education}
\cventry{Feb 2017 -- Sep 2020}{Master of Engineering}{Institut Teknologi Sepuluh Nopember}{Surabaya}{GPA: 3.91/4.00}{\textbf{Department of Electrical Engineering}, with \textbf{Multimedia Intelligent Network} Study Program.}
\cventry{Aug 2009 -- Sep 2015}{Bachelor of Engineering}{Institut Teknologi Sepuluh Nopember}{Surabaya}{GPA: 3.11/4.00}{\textbf{Department of Electrical Engineering}, focused in \textbf{Computer Engineering and Telematics}.}




\section*{Bachelor's Degree Final Project}
\cvitem{Title}{\emph{Virtual Machine Performance Measurement on Cloud Computing}}

\cvitem{Supervisors 1}{Mochamad Hariadi, ST., M.Sc., Ph.D.}

\cvitem{Supervisors 2}{Christyowidiasmoro, ST., MT.}

\cvitem{Description}{Built cloud servers to provide virtual machine (VM) as main services or IaaS with OpenStack. The VM was measured with benchmark programs to obtain the data about its performance metrics relative to its platform specifications (CPU, RAM and storage).}




\section{Master's Degree Final Project}
\cvitem{Title}{\emph{The Calculation of Player's and Non-Player Character's Gameplay Attribute Growth in Role-Playing Game with \textit{k}-NN and Naive Bayes.}}

\cvitem{Supervisors 1}{Prof. Dr. Ir. Mauridhi Hery Purnomo, M.Eng.}

\cvitem{Supervisors 2}{Dr. Supeno Mardi Susiki Nugroho, ST., MT.}

\cvitem{Description}{Created program to generate the RPG player character or enemies game attributes based on \textit{k}-NN and Naive Bayes algorithm then the result was classified with Neural Network Multi-class Classification. This project was created using Python programming language.}




\section*{Computer Skills}
\cvline{Languages}{C/C++, Python, Bash Shell Scripting.}

\cvline{Artificial Intelligence}{K-NN (Nearest Neighbor), Bayesian Method, Cluster Analysis, Regression Analysis, Classification Analysis, Decision Tree, Genetic Algorithm, Neural Network, Deep Learning.}

\cvline{Databases}{MySQL, MongoDB, Redis.}

\cvline{OS}{Debian, Ubuntu, CentOS, FreeBSD, OSX, Windows.}

\cvline{Network and Linux Operations}{Cisco and Huawei Hardware (Router and Switch) Operation, Mikrotik Router, Linux Network Operations (Gateway and Other Servers), Linux Commands Shell, Cloud Computing, OpenStack Cloud Framework, Virtualization (VirtualBox, KVM, and VMware).}

%\cvline{Project Man.}{Agile Project Management.}

\cvline{Office Tools}{\LaTeX, LibreOffice, Ms. Word, Ms. Excel, Ms. Power Point.}

\cvline{Simulator}{Matlab, CircuitMaker, Cisco Packet Tracer, Huawei eNSP.}

\cvline{Misc.}{HTML, CSS, Bootstrap Framework, Git, Open Computer Vision Library (OpenCV), Computer Hardware, Computer Performance Benchmarking, Software Development Life Cycle (SDLC), Agile Project Management.}




\section*{Achievements}
\cventry{May 2012}{1\textsuperscript{st} Place on ITS Hacking Competition 2012}{Institut Teknologi Sepuluh Nopember}{}{Surabaya}{This competition purposed to got the vulnerability of the one of servers in ITS.}

\cventry{April 2013}{2\textsuperscript{nd} Place on GKPKM (Gelar Karya Program Kreatifitas Mahasiswa) ITS EXPO 2013}{Institut Teknologi Sepuluh Nopember}{}{Surabaya}{This competition was similar or preparation for national science fair but at internal university level.}

\cventry{Sept 2013}{Finalist on PIMNAS (Pekan Ilmiah Mahasiswa Nasional) XXVI}{Research, Technology and Education Ministry}{Mataram University}{Mataram}{This competition was indonesian national science fair for undergraduate university student.}




%\section*{Training}
%\cventry{Nov 2009}{LKMM Pra--TD (Pre--Basic Student Management Skill Training)}{Faculty of Industrial Technology}{Institut Teknologi Sepuluh Nopember}{Surabaya}{}

%\cventry{May 2010}{LKMM TD (Basic Student Management Skill Training)}{Department of Electrical Engineering}{Institut Teknologi Sepuluh Nopember}{Surabaya}{}

%\cventry{May 2014}{Big Data Workshop}{Telematics Laboratory (B.201), PT. Solusi 247 and PT. XL Axiata Tbk}{Institut Teknologi Sepuluh Nopember}{Surabaya}{}

%\cventry{Nov 2014}{Huawei Certified Datacom Associate (HCDA)}{Huawei Technologies Co. Ltd.}{PUSTIKNAS (National ICT Center)}{South Tangerang}{}




\section*{Certification}
\cventry{Nov 2014}{Huawei Certified Network Associate (HCNA)}{Huawei Technologies Co. Ltd.}{PUSTIKNAS (National ICT Center)}{South Tangerang}{Credential ID: 010200100495806019171617}
\cventry{Sept 2020}{Agile Crash Course: Agile Project Management; Agile Delivery}{Udemy}{Online}{}{Credential ID: \href{https://www.udemy.com/certificate/UC-855d2519-7f1a-4a4f-9019-6dac452ab59b/}{UC-855d2519-7f1a-4a4f-9019-6dac452ab59b}}




%\section*{Organization -- Social Experience}
%\cventry{Nov 2010}{International Electrical Engineering Expo (IEEE)}{Electrical Engineering Department}{Institut Teknologi Sepuluh Nopember}{Surabaya}{As committee of sub-event LCEN (National Electronic Invention Competition)}

%\cventry{July 2010 -- Dec 2011}{BEM (Badan Eksekutif Mahasiswa) ITS}{Institut Teknologi Sepuluh Nopember}{}{Surabaya}{As Student Resource Development Staff}

%\cventry{Sept 2010}{GERIGI (Generasi Integralistik) ITS 2010}{Institut Teknologi Sepuluh Nopember}{}{Surabaya}{As Organizing Committee Coordinator.}

%\cventry{Sept 2011}{GERIGI (Generasi Integralistik) ITS 2011}{Institut Teknologi Sepuluh Nopember}{}{Surabaya}{As Steering Committee Coordinator.}




\section*{Publications}
\cventry{IEEE}{F. Tsabita, W. N. Rohman, Rosmaliati, B. P. V. Lystianingrum and M. H. Purnomo,}{``Semi-Supervised Learning Optimization Based on Generative Models to Identify Type Of Electric Load at Low Voltage''}{2018 International Seminar on Intelligent Technology and Its Applications (ISITIA)}{Bali, Indonesia, 2018}{pp. 209-214. doi: 10.1109/ISITIA.2018.8711235}

\cventry{IEEE}{N. R. Widiyanto, S., Nugroho, S. M. S., and M. H. Purnomo,}{``The Calculation of Player's and Non-Player Character's Gameplay Attribute Growth in Role-Playing Game with K-NN and Naive Bayes''}{2020 International Conference on Computer Engineering, Network, and Intelligent Multimedia (CENIM)}{Surabaya, Indonesia, 2020}{Status: ACCEPTED}




\section{Related Projects}
\cvline{\textbf{Cygnus Gateway}}{Built a computer gateway using Debian, Ubuntu or FreeBSD to provide internet access in Telematics Laboratory (B.201).}

\cvline{\textbf{Kirby File Sharing}}{Built a computer sharing server using Ubuntu to provide filesharing system in Telematics Laboratory (B.201) with FTP, Samba and NFS protocol.}

%\cvline{\textbf{ARvertise-ment}}{An integration of technology in advertisement using Augmented Reality in Android devices. This project was also funded by Ministry of Research, Technology and Higher Education. (URL Link: \url{youtu.be/xcrPnTn4uZU}).}

\cvline{\textbf{Redis Benchmark}}{Benchmark Redis loads with random content using Node.js and bash scripting to measure memory requirement and database down behavior.}

\cvline{\textbf{Database Backup}}{Created an automatic scheduled backup database program or scripts to backup MongoDB and Redis Databases from AWS (Amazon Web Service) to local server using CRON and Bash Shell scripting. This script was also used or edited to support MySQL or MariaDB Database backup.}

%\cvline{\textbf{Don't Die!}}{A simple game about saving the dying patient with electrocardiogram. This game was created during Global Game Jam 2017 event alongside my friends. (URL Link: \url{globalgamejam.org/2017/games/dont-die}).}

%\cvline{\textbf{Deblurring Image}}{An assignment from Genetic Algorithm class in Masters degree, about deblurring an image with genetic algorithm implementation. This project was created used MATLAB. (URL Link: \url{github.com/rohwid/debluring-image-genetic-algorithm}).}

%\cvline{\textbf{Stereo Camera Color Tracking}}{Created a program to detect an object with specific colors using HSV filter and built a DIY stereo camera from two ordinary webcams. The program will show the measurement result or distance from the camera to the object after the calibration process to setup the cameras to be used as stereo camera. This project was created used C++ with the OpenCV library. (URL Link: \url{github.com/rohwid/multi-object-color-tracking-stereo}).}

%\cvline{\textbf{EMG identification with K-NN and Naive Bayes}}{Created a program to identify the EMG (Electromyography) as a biometrical feature from the human. The program read and classify the datasets from human EMG and the name as the label. When it’s given an input about human EMG data, the program will be classified who has the EMG data as a result. The EMG data from people that became an input came from the same people in the datasets, it separated about 70\% for training and 30\% for testing. This project was created used MATLAB. (URL Link: \url{github.com/rohwid/emg-classification}).}

\cvline{\textbf{OpenStack IaaS Deployment Script}}{Built a cloud computing solution to provide IaaS (infrastructure as a Service) in laboratory. Used 5 computer server nodes (1 Controller and 4 Computes) to provide IaaS with the services such as Keystone (Authentication service), Glance (Image service), Nova (Compute service), Cinder (Block storage service), Horizon (User interface). The script for this project was also designed to make the deployment process easier and scalable (if want to add more servers). (URL Link: \url{github.com/rohwid/openstack-iaas-deploy}).}

\cvline{\textbf{PACS Server with ORTHANC}}{An assignment from Telemedicine class in master degree, about implementing an PACS (Picture Archiving and Communication System) server used ORTHANC to save DICOM (Digital Imaging and Communications in Medicine) images. The server may also be accessed from DICOM Viewer Software.}

\cvline{\textbf{Auto Nvidia Cuda Driver}}{A script to install NVIDIA drivers, CUDA, CUDNN and NCCL automatically in Linux to make deployment process more faster before do the deep learning training process with GPU. (URL Link: \url{github.com/rohwid/auto-nvidia-cuda-driver}).}

%\cvline{\textbf{Emotion Recognition}}{Create the emotion detection model with VGG16. The goal was to detect 7 emotions (neutral, happy, sad, surprise, angry, fear and disgusted) but in this project only 5 emotion (neutral, happy, sad, surprise, and angry) which stable enough when detect the people emotions. (URL Link: \url{github.com/rohwid/emotion-recognition}).}




%\section{References}
%\cventry{}{Marc Poisson}{VISIMMO 3D}{Paris}{marc.poisson@v3d-corporate.com}{Co-fondateur et associé. Directeur des pôles V3D Immo, V3D Com, V3D Formation.}
%\cventry{}{David Thibault}{Adways}{Lyon}{dthibaul@gmail.com}{Développeur 3D.}

\end{document}
